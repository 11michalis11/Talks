\documentclass{article}

\title{A 3-player game theoretic model of a choice between two queueing systems with strategic managerial decision making}
\author{Michalis Panayides}

\begin{document}

\maketitle
\begin{abstract}

    The main focus of this study is the construction of a 3-player game 
    theoretic model between two queueing systems and a service that distributes 
    individuals to them. 
    The resultant model is then used to explore dynamics between all players.

    The first aspect of this work is the development of a queueing system
    with two waiting spaces and two types of individuals. 
    Two modelling techniques were used to construct such a model; discrete event 
    simulation and Markov chains. 
    The state probabilities of the Markov chain system have been used to extract
    the performance measures of the queueing model (e.g. mean time in each 
    waiting room, mean number of individuals in each room, etc.)
    Moreover, the scenario that is explored is when two such systems
    exist, and we aim to distribute individuals among the two such that certain
    performance measures are optimised. 
    A 3-player game theoretic model is proposed between the two
    queueing systems and the service that distributes individuals to them. 
    In particular this can be seen as a 2-player normal form game where the 
    utilities are determined by a third player with its own strategies and 
    objectives. 
    A backwards induction technique is used to get the utilities of the normal 
    form game between the two queueing systems.

    This study's aim is to explore a queueing theoretic model with 
    two waiting spaces and use it to inform a game theoretic system. This 
    particular system can be applied in a healthcare scenario where it could
    capture the emergent behaviour between the Emergency Medical Service and the
    Emergency Department. The model will be used to form a 3-player game between
    the EDs of two neighbouring hospitals and the EMS.
    
\end{abstract}
    
\end{document}

