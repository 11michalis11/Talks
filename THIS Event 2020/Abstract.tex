\documentclass{article}

\title{A game theoretic model of the behavioural gaming that takes place at the EMS - ED interface}
\author{Michalis Panayides}

\begin{document}

\maketitle
\begin{abstract}
    Mathematical modelling can be utilised to model a vast range of scenarios
    applied to areas like healthcare, supermarkets and many more. Although, such 
    methods are really powerful, they sometimes lack the possible insight that one 
    may get by considering more qualitative approaches. The \textit{mother} of 
    such approaches is considered by many to be ethnography. 

    Ethnographic studies have been around since the \(19^{th}\) century and
    their main ideology is that a researcher that aims to study a specific social 
    setting should acquire a deeper understanding of a culture's norms and values. 
    The main technique that accompanies ethnography is participant observation, 
    where the researcher participates in the social setting they wish to study and 
    also records any observations made on it. Two additional concepts of 
    ethnography are the ideas of emic models and etic models. An etic model is 
    built based solely on the knowledge of the modeller and how they perceive the 
    model. On the contrary an emic model takes into consideration the perspective 
    of the individuals that belong in the social setting of interest. Ethnography 
    suggests that an emic model should always be the default type of model used 
    when studying any kind of social setting.

    A great application of these ideas and principles is the emergent behaviour
    that takes place on the interface between Emergency Medical Services (EMS) and 
    the Emergency Department (ED). Numerous decisions are taken by both patients 
    and staff alike that determine the level of workflow and the patient pathway 
    daily. There is empirical evidence to suggest that imposing targets in the ED 
    results in gaming at the interface of care between the EMS and ED. Multiple 
    scenarios are examined where an ambulance service needs to distribute patients 
    between neighbouring hospitals. The interaction between the hospitals and the 
    ambulance service is modelled in a game theoretic framework where the 
    ambulance service has to decide how many patients to distribute to each 
    hospital.
        
\end{abstract}
    
\end{document}
