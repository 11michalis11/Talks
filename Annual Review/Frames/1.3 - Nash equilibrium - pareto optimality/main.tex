% Slide 1
\begin{frame}
    \frametitle{Nash Equilibrium}
    \centering
    \begin{equation*}
        \begin{bmatrix}
            (3,3) & (0,5) \\
            (5,0) & (1,1)
        \end{bmatrix}
    \end{equation*}

    \tikz[overlay] {
        \node (arrow_1_end) at (-1.5, 1.2) {\Large{C}};
        \node (arrow_2_end) at (-1.5, 0.7) {\Large{D}};
        \node (arrow_1_end) at (-0.6,1.8) {\Large{C}};
        \node (arrow_2_end) at (0.6,1.8) {\Large{D}};
        }
\end{frame}


% Slide 2
\begin{frame}
    \frametitle{Nash Equilibrium}
    
    \centering
    
    \begin{equation*}
        \begin{bmatrix}
            (3,-) & (0,-) \\
            (\underline{5},-) & (\underline{1},-)
        \end{bmatrix}
    \end{equation*}

    \tikz[overlay] {
        \node (arrow_1_start) at (-3, 1.2) {\Large{C}};
        \node (arrow_1_end) at (-1.5, 1.2) {};

        \node (arrow_2_start) at (-3, 0.7) {\Large{D}};
        \node (arrow_2_end) at (-1.5, 0.7) {};

        \draw[->, very thick, red] (arrow_1_start) -- (arrow_1_end);
        \draw[->, very thick, red] (arrow_2_start) -- (arrow_2_end);
        }
\end{frame}


% Slide 3
\begin{frame}
    \frametitle{Nash Equilibrium}
    
    \centering
    
    \begin{equation*}
        \begin{bmatrix}
            (-,3) & (-,\underline{5}) \\
            (-,0) & (-,\underline{1})
        \end{bmatrix}
    \end{equation*}

    \tikz[overlay] {
        \node (arrow_1_start) at (-0.6,3) {\Large{C}};
        \node (arrow_1_end) at (-0.6,1.5) {};

        \node (arrow_2_start) at (0.6,3) {\Large{D}};
        \node (arrow_2_end) at (0.6,1.5) {};

        \draw[->, very thick, red] (arrow_1_start) -- (arrow_1_end);
        \draw[->, very thick, red] (arrow_2_start) -- (arrow_2_end);
        }
\end{frame}


% Slide 4
\begin{frame}
    \frametitle{Nash Equilibrium}
    
    \centering
    \begin{equation*}
        \begin{bmatrix}
            (3,3) & (0,\underline{5}) \\
            (\underline{5},0) & (\underline{1},\underline{1})
        \end{bmatrix}
    \end{equation*}

    \tikz[overlay] {
        \node (arrow_1_start) at (-3, 0.7) {\Large{D}};
        \node (arrow_1_end) at (-1.5, 0.7) {};

        \node (arrow_2_start) at (0.6,3) {\Large{D}};
        \node (arrow_2_end) at (0.6,1.5) {};

        \draw[->, very thick, red] (arrow_1_start) -- (arrow_1_end);
        \draw[->, very thick, red] (arrow_2_start) -- (arrow_2_end);
        }
\end{frame}


% Slide 5
\begin{frame}
    \frametitle{Pareto Optimality}
    
    \centering
    \begin{equation*}
        \begin{bmatrix}
            (3,3) & (0,5) \\
            (5,0) & (1,1)
        \end{bmatrix}
    \end{equation*}

    \vspace{-0.5cm}
    \begin{equation*}
        \overbrace{(3,3), \, (0,5), \, (5,0), \, (1,1)}
    \end{equation*}
\end{frame}


% Slide 6
\begin{frame}
    \frametitle{Pareto Optimality}
    
    \centering
    \begin{equation*}
        \begin{bmatrix}
            (3,3) & (0,5) \\
            (5,0) & (1,1)
        \end{bmatrix}
    \end{equation*}

    \vspace{-0.5cm}
    \begin{equation*}
        \overbrace{(3,3), \, (0,5), \, (5,0), \, (1,1)}
    \end{equation*}

    \begin{equation*}
        (\underline{3},\underline{3}) > (1,1)
    \end{equation*}
\end{frame}